% !TEX TS-program = xelatex
\documentclass[FM]{tulpresentation} % https://github.com/RadekMocek/tulpresentation
\usepackage{listings}
\definecolor{color_gray}{rgb}{0.5294117647058824, 0.5803921568627451, 0.5764705882352941}
\definecolor{color_violet}{rgb}{0.403921568627451, 0.4196078431372549, 0.7607843137254902}
\definecolor{color_blue}{rgb}{0.13725490196078433, 0.49019607843137253, 0.7450980392156863}
\definecolor{color_green}{rgb}{0.4549019607843137, 0.5215686274509804, 0.00784313725490196}
\lstset{
    , basicstyle = \ttfamily\bfseries
    , breaklines = false
    , frame = none
    , commentstyle = \color{color_gray}
    , keywordstyle = \color{color_violet}
    , ndkeywordstyle = \color{color_blue}
    , stringstyle = \color{color_green}
    , showstringspaces = false
    , prebreak = \raisebox{0ex}[0ex][0ex]{\ensuremath{\hookleftarrow}}
}

\lstdefinelanguage{pseudo}{
  , keywords={fn, new, if, static}
  , ndkeywords={init, str, add_widget, connect, clicked, button_clicked, set_text, update, btn}
}
\title{Grafické uživatelské rozhraní v režimu okamžitého vykreslování}
\titlesub{Radek Mocek}{}{Jiří Jeníček, Ph.D.}{}
\begin{document}
	% "titulní slajd: název práce, vaše jméno, jméno vedoucího"
	\TULtitleframe
	
	% "druhý slajd: zadání"
	\begin{frame}
		\frametitle{Cíle diplomové práce}
		\begin{enumerate}
			\item Proveďte rešerši GUI v režimu okamžitého vykreslování.
			\item Navrhněte ukázkovou aplikaci.
			\item Ve vybraných knihovnách implementujte a otestujte aplikaci.
			\item Zhodnoťte výsledky.
		\end{enumerate}
	\end{frame}

	% "teoretický úvod – popis řešené problematiky"
	\begin{frame}
		\subheader{RMGUI – Retained Mode GUI}
		\begin{itemize}
			\item „Režim soběstačného vykreslování“
			\item V inicializační fázi se definují jednotlivé ovládací prvky a na ně se navěšují callback metody
			\item Hlavní smyčka pak řeší vykreslování prvků a reakce na události, před programátorem je skryta
			\item Knihovny \textit{Qt}, \textit{wxWidgets}, \textit{WPF}, \dots
		\end{itemize}
		\subheader{\\IMGUI – Immediate Mode GUI}
		\begin{itemize}
			\item „Režim okamžitého vykreslování“
			\item Žádná inicializační fáze ani callback metody, pouze hlavní smyčka
			\item Definice zmiňuje pouze tento rozdíl z pohledu programátora koncové aplikace, implementace knihoven mohou být různé
			\item Knihovny \textit{Dear ImGui}, \textit{egui}, \textit{Nuklear}, \dots
		\end{itemize}
	\end{frame}
	
	\begin{frame}[fragile]
	\subheader{RMGUI}
	\lstinputlisting[language=pseudo]{code/rmgui.txt}
	\subheader{IMGUI}
	\lstinputlisting[language=pseudo]{code/imgui.txt}
	\end{frame}
	
	\begin{frame}
		\frametitle{IMGUI v praxi}
		\begin{itemize}
			\item Využití grafického API pro vykreslování ovládacích prvků (\textit{OpenGL}, \textit{DirectX}, \dots)
			\item Časté použití pro přidání ovládacích prvků do existující interaktivní aplikace
			\item Spíše GUI pro ladění vývojářem, než user-facing GUI
			\item Chaos
			\begin{itemize}
				\item Nepříliš striktní definice, různé interpretace
				\item Přívlastek „okamžitý“ vybrán z důvodu určité podobnosti k \textit{immediate-mode graphics API}, není to ale totéž
				\item Název knihovny \textit{Dear ImGui} se často zkracuje na \textit{ImGui},\\ ale \textit{ImGui} $\neq$ \textit{IMGUI}
			\end{itemize}
		\end{itemize}
	\end{frame}
	
	\begin{frame}
		\subheader{Motivace}
		\begin{itemize}
			\item Jak jsou IMGUI knihovny použitelné pro psaní „běžných“ user-facing aplikací?
			\item Jaký je dopad na výkon oproti RMGUI knihovnám?
			\item IMGUI knihovny jsou v aktivním vývoji
			\item Vnést do chaosu trochu řádu
		\end{itemize}
		\subheader{Ukázková aplikace}
		\begin{itemize}
			\item WYSIWYM diagram editor
			\item Implementováno v \textit{Dear ImGui}, \textit{egui} a \textit{Qt}
			\item Momentálně: Obdélníky s textem, které lze spojovat šipkami
			\item TODO: Spíše než zlepšovat editor diagramů se soustředit na schopnosti daných knihoven: syntax highlight, pás tlačítek s nástroji, klávesové zkratky, systémové dialogy, nastavení vzhledu, \dots
		\end{itemize}
	\end{frame}
	
	% "představa vlastního obsahu práce"
	\begin{frame}
		\includegraphics[width=.95\paperwidth]{img/dp_dear_imgui}
	\end{frame}
	
	% Děkuji
	\TULendframe
\end{document}
