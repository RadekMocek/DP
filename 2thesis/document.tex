% !TEX TS-program = xelatex

% Inicializace tulthesis
\documentclass[FM,DP]{tulthesis}
% Typografie pro češtinu; xelatex alternativa pro babel
\usepackage{polyglossia}
\setdefaultlanguage{czech}
% Angličtina pro abstrakt
\makeatletter
\ExplSyntaxOn
\pretocmd\xpg@set@alias@values{%
	\prop_if_exist:cF { xpg@alias@keyvals@#1@#4 }
	{ \prop_new:c {xpg@alias@keyvals@#1@#4} }
}{}{}
\ExplSyntaxOff
\makeatother
\setotherlanguage{english}
% Automatické pevné mezery
\usepackage{xevlna}
% Zabránění vdov a sirotků
\usepackage[all]{nowidow}
% Snížit šanci na rozdělení posledního slova odstavce
\finalhyphendemerits=200000
% Díky balíku caption kliknutí na \ref odkazující na obrázek skočí opravdu na obrázek, a ne na popisek pod ním
\usepackage{caption}
% Tabulky
\usepackage{enumitem} % Umožní prostředí NiceTabular používat vlastní footnotes
\usepackage{nicematrix} % Posyktuje prostředí NiceTabular

% Pro seznam použité literatury
\usepackage[backend=biber, style=iso-numeric]{biblatex}
\addbibresource{zdroje.bib}
% Poslední dva autory knihy neoddělovat středníkem ale písmenem 'a'
\DeclareDelimFormat{finalnamedelim}{\addspace a\space}

% Název práce
\TULtitle{Grafické uživatelské rozhraní v režimu okamžitého vykreslování}{Graphical user interface in immediate rendering mode}

% -- Prozatimní úvodní stránka --
\TULprogramme{N0613A140028}{Informační technologie}{Information technology}
\TULbranch{}{Aplikovaná informatika}{Applied Informatics}
\TULauthor{Radek Mocek}
\TULsupervisor{Ing. Jiří Jeníček, Ph.D.}
\TULyear{2025}
% -------------------------------

% Začátek dokumentu
\begin{document}
	
	\ThesisStart{male} % Prozatimní úvodní stránka
	% Úvodní stránky ze STAGu
	%\ThesisStart{vskp_-_zadani_vskp.pdf}	
	
	% Poděkování
	\begin{acknowledgement}
		Děkuji.
	\end{acknowledgement}
	
	% Abstrakt česky
	\begin{abstractCZ}
		Abstrakt česky.
	\end{abstractCZ}
	
	% Klíčová slova česky
	\begin{keywordsCZ}
		Klíčová, slova, česky.
	\end{keywordsCZ}
	\clearpage
	
	% Abstrakt anglicky
	\begin{abstractEN}		
		\begin{english}
			English abstract.
		\end{english}
	\end{abstractEN}
	
	% Klíčová slova anglicky
	\begin{keywordsEN}
		\begin{english}English, keywords.\end{english}
	\end{keywordsEN}
	
	% Obsah
	\tableofcontents
	
	% Seznam obrázků
	\listoffigures
	
	% Seznam tabulek
	\listoftables
	
	\clearpage
	
	% Zkratky
	\begin{abbrList}
		\textbf{API} & application programming interface, rozhraní pro programování aplikací \\
		\textbf{GUI} & graphical user interface, grafické uživatelské rozhraní \\
		\textbf{IMGUI} & immediate mode graphical user interface, grafické uživatelské rozhraní v režimu okamžitého vykreslování \\
		\textbf{RMGUI} & retained mode graphical user interface, grafické uživatelské rozhraní v režimu soběstačného vykreslování \\
		\textbf{WPF} & Windows Presentation Foundation, knihovna umožňující tvorbu GUI aplikací pro operační systém Windows \\
	\end{abbrList}
	
	% Jádro zprávy
	\chapter{Úvod}
	
	Úvod
	
	\chapter{Grafické uživatelské rozhraní v režimu okamžitého vykreslování}
	
	Grafické uživatelské rozhraní v režimu okamžitého vykreslování (\textit{immediate mode graphical user interface}, zkráceně IMGUI) je specifický přístup k programování aplikačního softwaru s grafickým uživatelským rozhraním. Základní myšlenku tohoto přístupu veřejně představil Casey Muratori již v roce 2005. Definice IMGUI není příliš komplexní ani striktní, a proto se postupem času začaly objevovat různé úhly pohledu a interpretace toho, co by „okamžitý režim“ v kontextu vývoje GUI aplikací měl a neměl znamenat. \cite{online_casey}
	
	Autor přístupu IMGUI využil pro jeho pojmenování přívlastek „okamžitý“ z důvodu určité podobnosti k režimu okamžitého vykreslování v některých grafických API té doby \cite{online_casey}. Příkladem může být Direct3D, grafické API od společnosti Microsoft, které mohlo dříve pracovat ve dvou režimech: soběstačný (\textit{retained}) a okamžitý (\textit{immediate}). Z pohledu abstrakce je soběstačný režim na vyšší úrovni, kdy si knihovna interně uchovává model 3D scény a od aplikačního kódu přijímá příkazy, které manipulují s objekty v této scéně. Pro vykreslení snímku pak knihovna transformuje informace o objektech ve scéně na sadu příkazů pro vykreslování. V okamžitém režimu se s modelem scény pracuje na straně aplikačního kódu a knihovna přijímá až samotné příkazy pro vykreslování. Tím sice narůstá počet úkonů, které musí být provedeny na straně aplikačního kódu, zároveň se ale zvětšuje flexibilita a prostor pro optimalizace. \cite{book_direct3dbible, online_direct2d}
	
	% [? Diagram IM vs RM, případně diagram dobového Direct3D ?]
	
	Podobně lze nahlížet na programování GUI aplikací. Běžně používané knihovny, které tvorbu GUI aplikací umožňují, jsou v této analogii chápány jako soběstačný režim a přísluší jim zkratka RMGUI (\textit{retained mode graphical user interface}). S využitím těchto knihoven definuje aplikační kód jednotlivé ovládací prvky, u kterých obvykle specifikuje jejich identifikátory, vzájemné rozložení, a případné callback metody. Knihovna pak ve své interní smyčce řeší vykreslování ovládacích prvků a reakce na události. Mezi zástupce tohoto přístupu patří např. knihovny Qt a WPF. Oproti tomu aplikace využívající knihovnu s přístupem IMGUI \dots
	
	% * Stav ovládacích prvků, ...
	% * na imgui pohlížíme z pohledu api, neznámená to tedy, že imgui knihovna nemůže pod kapotou mít uložený nějaký stav, a nebo že musí překreslovat celé gui každý frame
	% * i přesto jde ale ruku v ruce s realtime aplikacemi
	% * výhody
	% * nevýhody	
	
	\section{Demonstrace rozdílu mezi přístupy RMGUI a IMGUI}
	
	% * gui není jen imgui vs rmgui; viz webové aplikace, jetpack?, MVVM
	
	\section{Existující knihovny pro tvorbu GUI aplikací s přístupem IMGUI}
	
	% * i když je imgui starý koncept, knihovny jsou stále ve vývoji
		
	\begin{table}[ht]
		\centering
		\caption{Přehled knihoven pro tvorbu GUI aplikací s přístupem IMGUI}
		\begin{NiceTabular}{ l l l c r }[cell-space-limits=3pt]
			\CodeBefore
				\rowcolors[gray]{2}{0.95}{}
			\Body
				\RowStyle{\bfseries}
					Název & Jazyk & Licence & Vznik & Popularita\tabularnote[$\star$]{počet hvězd na platformě GitHub ke dni 2. 11. 2025} \\
				\hline
					Dear ImGui  & C++  & MIT             & 2014 & 69 088 \\
					egui        & Rust & MIT, Apache-2.0 & 2018 & 26 997 \\
					Nuklear     & C    & MIT, Unlicense  & 2015 & 10 522 \\
					microui     & C    & MIT             & 2018 &  5 117 \\
					raygui      & C    & Zlib            & 2014 &  4 407 \\\RowStyle[cell-space-bottom-limit=1pt]{}
					Gio         & Go   & MIT, Unlicense  & 2019 &  2 050 \\
				\hline
		\end{NiceTabular}
	\end{table}
	
	% Zdroje
	\chapter*{Seznam použité literatury}
	\addcontentsline{toc}{chapter}{Seznam použité literatury}
	\printbibliography[heading=none]
	
\end{document}
